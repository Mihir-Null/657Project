\section{\textbf{Potential Improvements}} \label{imp}
\par In this project, we focused exclusively on analyzing magic state distillation using the standard Bravyi–Haah protocol applied to weakly triply-even and related CSS code constructions. While this framework is well-established and provides a concrete mathematical baseline for evaluating distillation performance, our project could certainly incorporate more complex and efficient protocols.

\par Because our analysis relies entirely on the Bravyi–Haah protocol and its associated error-suppression behavior, our results do not reflect the most resource-efficient distillation strategies currently available. More recent approaches incorporate additional circuit optimizations, adaptive measurement strategies, and correlated error handling that can lead to substantially improved overhead and performance. For example, we explored the possibility of using a Monte Carlo–based distillation simulation instead of relying purely on the analytic Bravyi–Haah framework.

\par Such a simulation framework would make it possible to benchmark our doubled QR and weakly triply-even constructions against modern, state-of-the-art distillation schemes under practical operating conditions.

\par Finally, our work focused on distillation under idealized assumptions, including perfect Clifford operations and noiseless measurements. A natural extension would be to incorporate circuit-level noise models and analyze how error correlations propagate across multiple distillation rounds. This would significantly strengthen the practical relevance of our results for near-term fault-tolerant quantum computing.

\par Ultimately, a major direction for future improvement is the implementation of more efficient and realistic distillation protocols -- particularly Monte Carlo-based simulation strategies -- to obtain tighter, experimentally relevant performance estimates.
