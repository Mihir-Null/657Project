\section{\textbf{Potential Improvements}} \label{imp}
\par In this project, we focused exclusively on analyzing magic state distillation assuming a standard Bravyi-Haah protocol applied to the weakly triply-even and related CSS code constructions. While this framework is well-established and provides a concrete mathematical baseline for evaluating distillation performance, it does not reflect the most resource-efficient protocols (and optimized circuits) available on current hardware.\\
More recent distillation approaches also incorporate additional circuit optimizations, adaptive measurement strategies, and correlated error handling that can lead to substantially improved overhead and performance.


\par Our results likely underestimate the achievable performance of the code families we studied given that we performed theoretical asymptotic analysis and did not derive analytic closed bounds or compute yields at large block lengths.
\par Such a simulation framework would make it possible to directly estimate acceptance probabilities, conditional logical error rates, and overall yield under realistic stochastic noise models. This would enable direct benchmarking of our doubled QR and weakly triply-even constructions against modern state-of-the-art distillation schemes under practical operating conditions.

\par Finally, our analysis assumed idealized conditions, including near perfect Clifford operations and a simplistic noise model. A natural extension would be to incorporate circuit-level noise models on an actual simulated circuit and study how error correlations propagate across multiple rounds of distillation. This would significantly strengthen the practical relevance of our results for near-term fault-tolerant quantum computing.

\par Ultimately, the most important direction for future improvement is an analytic bound on the distillation yield and overhead as well as the implementation of a concrete distillation protocol