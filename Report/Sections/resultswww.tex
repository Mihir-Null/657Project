\section{\textbf{Results and What Went Well}} \label{www} 

\par Our results demonstrate that QR-based weakly triply-even (TE*) codes should exhibit strong logical error suppression at small physical error rates, consistent with our original hypothesis. In particular, preliminary simulations indicate that TE* constructions achieve high effective distance at relatively small block lengths, leading to a strong error-suppression exponent of the form $p_{\text{out}}(p) = O(p^d)$. This behavior supports the claim that QR-based TE* codes offer a practical advantage in the finite-size regime, rather than only asymptotic improvements.

\par The hypothesis that QR-based TE* codes outperform standard Bravyi-Haah constructions for realistic block sizes was supported by our initial simulation results. The strengths observed were the combination of high distance for small $n$, strong logical error mitigation, and the presence of transversal diagonal gates arising from the underlying divisibility conditions. 

\par Ultimately, while our results demonstrate strong finite-size performance for QR-based TE* codes, substantial opportunities remain to extend both the analytical and numerical scope of the project toward more realistic and scalable fault-tolerant quantum computing architectures.