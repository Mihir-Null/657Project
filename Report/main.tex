\documentclass[11pt]{amsart}
\usepackage{times}
\usepackage{cite}
\usepackage{booktabs}

\usepackage{pgfplots}
\usepackage{pgfplotstable}
\pgfplotsset{compat=newest}

\usepackage[algo2e,ruled,lined,linesnumbered,noend]{algorithm2e}
\usepackage[noend]{algpseudocode}

\usepackage[margin=1in,footskip=0.25in]{geometry}

% \usepackage{parskip}
% \setlength{\parindent}{0em}
% \parskip 0.75em
\usepackage{amsmath,amssymb,amsthm}


\usepackage{mathtools}
\usepackage{enumerate}
\usepackage{color}   %May be necessary if you want to color links

\usepackage{hyperref}
\hypersetup{
    colorlinks=true, %set true if you want colored links
    linktoc=all,     %set to all if you want both sections and subsections linked
    allcolors=blue,  %choose some color if you want links to stand out
}


\SetArgSty{textrm}
\SetKwProg{Fn}{Function}{}{}

\DeclareMathOperator{\poly}{polylog}
\DeclareMathOperator{\supp}{supp}
\DeclareMathOperator{\psupp}{psupp}
\DeclareMathOperator{\Fix}{Fix}

\newcommand{\res}{\upharpoonright} 

\newcommand{\E}{\mathbb{E}}
\newcommand{\mf}[1]{{\mbox{\sc{#1}}}}
\newcommand{\bb}[1]{\mathbb{#1}}
\newcommand{\mc}[1]{\mathcal{#1}}
\newcommand{\inv}{\mathrm{inv}}

\algdef{SE}[SUBALG]{Indent}{EndIndent}{}{\algorithmicend\ }
\algtext*{Indent}
\algtext*{EndIndent}

\numberwithin{equation}{section}

\theoremstyle{plain}
\newtheorem{thm}[equation]{Theorem}
\newtheorem{cor}[equation]{Corollary}
\newtheorem{lem}[equation]{Lemma}
\newtheorem{clm}[equation]{Claim}
\newtheorem{prop}[equation]{Proposition}

\theoremstyle{definition}
\newtheorem{defn}[equation]{Definition}
\newtheorem{fact}[equation]{Fact}

\theoremstyle{remark}
\newtheorem{rem}[equation]{Remark}
\newtheorem{ex}[equation]{Example}
\newtheorem{notation}[equation]{Notation}
\newtheorem{terminology}[equation]{Terminology}


\newtheorem{innercustomthm}{Lemma}
\newenvironment{clemma}[1]
  {\renewcommand\theinnercustomthm{#1}\innercustomthm}
  {\endinnercustomthm}

%%%-------------------------------------------------------------------
%%%-------------------------------------------------------------------
%%%-------------------------------------------------------------------
%%%-------------------------------------------------------------------
%%%-------------------------------------------------------------------
%%%-------------------------------------------------------------------
%%%-------------------------------------------------------------------


% \makeatletter
% \newcommand{\leqnomode}{\tagsleft@true\let\veqno\@@leqno}
% \newcommand{\reqnomode}{\tagsleft@false\let\veqno\@@eqno}
% \makeatother


\begin{document}

%%% In the title, use a double backslash "\\" to show a linebreak:
%%% Use one of the following two forms:
%%% \title{Text of the title}
%%% or
%%% \title[Short form for the running head]{Text of the title}
\title{CMSC657 Final Project Report}


%%% If there are multiple authors, they're described one at a time:
%%% First author: \author{} \address{} \curraddr{} \email{} \thanks{}
%%% Second author: \author{} \address{} \curraddr{} \email{} \thanks{}
%%% Third author: \author{} \address{} \curraddr{} \email{} \thanks{}
\author{Mihir Talati}
\author{Leo S.P. Velloso}


\maketitle

%%% To include a table of contents, uncomment the following line:

%\tableofcontents
% \newpage


\section{\textbf{Abstract}} \label{sec abstract} 

\par Magic state distillation is a key technique in fault-tolerant quantum computing that allows the implementation of non-Clifford gates -- namely the T gate -- using only operations that are, themselves, fault-tolerant under stabilizer codes. Stabilizer codes can perform Clifford gates: \textsc{Hadamard}, \textsc{CNOT}, and \textsc{P}, reliably and efficiently. However, Clifford gates alone are not universal for quantum computation. A non-Clifford gate is required to achieve full computational power and finding stabilizer codes that implement such gates fault tolerantly are rare. Moreover, both codes must have good distance (number of detectable errors) for the resources/qubits used.
\\  \\
\par The \textbf{Easten-Knill Theorem} states that no singular quantum error correcting code can implement both the Clifford group of gates and a Non-Clifford gate transversally (i.e fault tolerantly) simultaneously. Magic state distillation solves this problem by preparing many noisy copies of a special quantum state called a \textit{\textbf{magic state}}. This allows one to use Clifford operations and measurements to purify these states, detecting and discarding those affected by noise. \\ \\
\par Through iterative rounds of this purification, a small number of high-fidelity magic states are obtained, which can then be used in the computation to simulate fault-tolerant \textsc{T} gates. Although resource-intensive - requiring large overhead in qubits and operations - magic state distillation is currently the most practical method to achieve universal, error-corrected quantum computation. However, not all codes have equal yields for creating these \textit{\textbf{magic state}}, and the utility of many qubit and prime qudit codes is derived from the fidelity and efficacy with which they can be used to distill magic states.
\\ \\
\par In our project, we explored magic state distillation yields for families of doubled quadratic residue (QR) based \textsc{CSS} codes as constructed by Jain and Albert. There are three overlapping families described: doubly even QR \textsc{CSS} codes, weak triply even codes obtained via doubling QR codes and, triorthogonal codes obtained by doubling self dual codes. The latter two codes are promising candidates for a Bravyi-Haah style triorthogonal block distillation protocol, as the weak triply even codes requires no clifford corrections and both provide high order error suppression at high distances.
For multiple qubit diagonal circuits (many \textsc{CCZ} gates, adders, etc.) the same codes can also be used in Campbell-Howard synthillation.



\section{\textbf{Approach}} \label{sec approach} 
The following algorithm describes a naive brute force method to approximate per code distillation yields (good/faulty states out per noisy state in):   
\begin{algorithm2e}
\caption{Distillation$(H_X, z_{\log}, w_{\max})$}
\textbf{Input:} $H_X$ X-stabilizer matrix, $z_{\log}$ logical $Z$ vector, maximum weight $w_{\max}$ \\
\textbf{Output: } $w_{\min}$ (harmful undetected weight), $A_{w_{\min}}$ (multiplicity) \\[4pt]

$w_{\min} \gets \text{None}$
$A_{w_{\min}} \gets 0$

\For{$w \in \{1, \ldots, w_{\max}\}$}{
  $\text{\textsc{found}} \gets \text{False}$ \\
  \For{each subset $S \subseteq \{1,\dots,n\}$ with $|S| = w$}{
    Construct $e \in \{0,1\}^n$ by setting $e_i = 1$ iff $i \in S$
    \If{$H_X e^T = 0$}{ 
      \If{$z_{\log} \cdot e = 1$}{
        $\textsc{found} \gets \text{True}$ \\
        \If{$w_{\min} = \text{None}$}{
          $w_{\min} \gets w$\\ 
          $A_{w_{\min}} \gets 1$
        }
        \ElseIf{$w = w_{\min}$}{
          $A_{w_{\min}} \gets A_{w_{\min}} + 1$
        }
      }
    }
  }
  \If{\textsc{found}}{
    \textbf{break}
  }
}
\end{algorithm2e}
\newpage

\section{\textbf{Hypothesis}} \label{hypo} 
{\clm{
\textbf{Weakly triply-even} (\textsc{TE*}) codes derived from quadratic-residue constructions will achieve strictly better \emph{finite-size overhead} than both generic \textbf{doubled self-dual} constructions and the standard \textbf{Bravyi--Haah triorthogonal} distillation codes due to their high distances in $n$-qubit codes for small $n$, structure inherited from QR code weight distributions, and exceptionally low-weight X-stabilizers while still satisfying necessary triorthogonality conditions.
}}
\section{\textbf{Results and What Went Well}} \label{www} 

\par Our results demonstrate that QR-based weakly triply-even (TE*) codes should exhibit strong logical error suppression at small physical error rates, consistent with our original hypothesis. In particular, preliminary simulations indicate that TE* constructions achieve high effective distance at relatively small block lengths, leading to a strong error-suppression exponent of the form $p_{\text{out}}(p) = O(p^d)$. This behavior supports the claim that QR-based TE* codes offer a practical advantage in the finite-size regime, rather than only asymptotic improvements.

\par The hypothesis that QR-based TE* codes outperform standard Bravyi-Haah constructions for realistic block sizes was supported by our initial simulation results. The strengths observed were the combination of high distance for small $n$, strong logical error mitigation, and the presence of transversal diagonal gates arising from the underlying divisibility conditions. 

\par Ultimately, while our results demonstrate strong finite-size performance for QR-based TE* codes, substantial opportunities remain to extend both the analytical and numerical scope of the project toward more realistic and scalable fault-tolerant quantum computing architectures.
\section{\textbf{Potential Improvements}} \label{imp}
\section{\textbf{Contributions}} \label{sec contributions} 

\newpage




\nocite{*}
\bibliographystyle{plain}
\bibliography{refs}
\end{document}
