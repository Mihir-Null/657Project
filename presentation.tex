\documentclass[aspectratio=169]{beamer}
\usetheme{metropolis}

\setbeamertemplate{section in toc}{\textbullet\ \inserttocsection}
\setbeamertemplate{subsection in toc}{\hspace*{1.5em}\parbox[t]{0.75\textwidth}{\raggedright\footnotesize\textendash\ \inserttocsubsection}}

\title{Magic State Distillation From Quadratic Residue based CSS codes}
\subtitle{Placeholder Subtitle}
\author{Mihir Talati, Leonardo S.P Velloso}
\date{\today}

\begin{document}

\maketitle

\section{Presentation Outline}
\begin{frame}{Presentation Outline}
  \begin{columns}[t,onlytextwidth]
    \column{0.48\textwidth}
      \tableofcontents[sections={2-3},subsectionstyle=show/show,subsubsectionstyle=hide]
    \column{0.48\textwidth}
      \tableofcontents[sections={4-6},subsectionstyle=show/show,subsubsectionstyle=hide]
  \end{columns}
\end{frame}

\section{Background}
\subsection{Magic State Distillation Overview}
\begin{frame}{Magic State Distillation Overview}
  \begin{itemize}
    \item Placeholder overview of why magic state distillation is required for fault-tolerant quantum computing.
    \item Outline key challenges addressed by distillation in this study.
  \end{itemize}
\end{frame}

\subsection{Transversal Gates}
\begin{frame}{Transversal Gatesets Background}
  \begin{itemize}
    \item Placeholder summary of transversal gate properties and their role in error mitigation.
    \item Notes on constraints imposed by transversal operations on encoded logical gates.
  \end{itemize}
\end{frame}

\subsection{CSS Codes}
\begin{frame}{CSS Code Formalization}
  \begin{itemize}
    \item Placeholder notes on constructing CSS codes from classical linear codes.
    \item Mention error detection and correction properties relevant to distillation.
  \end{itemize}
\end{frame}

\subsection{Self-Dual Codes}
\begin{frame}{Self-Dual Codes}
  \begin{itemize}
    \item Placeholder explanation of self-duality criteria in the CSS setting.
    \item Comments on why self-dual structures matter for transversal gate compatibility.
  \end{itemize}
\end{frame}

\subsection{Doubled QR Codes}
\begin{frame}{Code Structure and Construction}
  \begin{itemize}
    \item Placeholder summary of the code family described in \texttt{https://arxiv.org/pdf/2408.12752}.
    \item Highlight key construction steps and parameters to emphasize during the talk.
  \end{itemize}
\end{frame}

\subsection{Distillation Protocol}
\begin{frame}{Bravyi-Haah Magic State Distillation}
  \begin{itemize}
    \item Placeholder outline of the protocols evaluated with the selected codes.
    \item Notes on assumptions, inputs, and targeted logical states.
  \end{itemize}
\end{frame}

\section{Motivation and Approach}
\subsection{Yield}
\begin{frame}{Distillation Yield}
  \begin{itemize}
    \item Placeholder definition of yield metrics used for comparison.
    \item Clarify normalization and resource accounting assumptions.
  \end{itemize}
\end{frame}

\subsection{Scaling}
\begin{frame}{Scaling With Code Length}
  \begin{itemize}
    \item Placeholder observations on how yield changes with increasing code size.
    \item Notes on asymptotic behavior and finite-size considerations.
  \end{itemize}
\end{frame}

\subsection{Methodology}
\begin{frame}{Bravyi-Haah Protocol for TE* and triorthogonal codes}
  \begin{itemize}
    \item Placeholder recap of triorthogonal code requirements and procedure.
    \item Points to connect the protocol to the chosen code family.
  \end{itemize}
\end{frame}

\subsection{Hypothesis}
\begin{frame}{Hypothesis}
  \begin{itemize}
    \item Placeholder list of tweaks made to the standard protocol.
    \item Rationale for modifications and anticipated impact on performance.
  \end{itemize}
\end{frame}

\section{Simulation}
\subsection{Algorithm}
\begin{frame}{Algorithm Overview}
  \begin{itemize}
    \item Placeholder description of the simulation setup and noise models.
    \item Mention computational tools or libraries anticipated for the study.
  \end{itemize}
\end{frame}

\subsection{Results}
\begin{frame}{Pretty Graphs}
  \begin{itemize}
    \item Placeholder description of the simulation setup and noise models.
    \item Mention computational tools or libraries anticipated for the study.
  \end{itemize}
\end{frame}

\subsection{Analysis}
\begin{frame}{Contextualizing Results}
  \begin{itemize}
    \item Placeholder interpretation of simulated performance metrics.
    \item Discussion points comparing outcomes to expectations or baselines.
  \end{itemize}
\end{frame}

\section{Conclusion}
\begin{frame}{HYpothesis vs Results}
  \begin{itemize}
    \item Placeholder statement of the working hypothesis before analysis.
    \item Criteria used to judge success or failure.
  \end{itemize}
\end{frame}

\section{Potential Future Work}
\begin{frame}{What's next?}
  \begin{itemize}
    \item Placeholder list of follow-up experiments and protocol refinements.
    \item Suggestions for code design or distillation strategy improvements.
  \end{itemize}
\end{frame}

\end{document}
