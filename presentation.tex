\documentclass[aspectratio=169]{beamer}
\usetheme{metropolis}

\title{Magic State Distillation Study}
\subtitle{Placeholder Subtitle}
\author{Your Name}
\date{\today}

\begin{document}

\maketitle

\section{Presentation Outline}
\begin{frame}{Presentation Outline}
  \tableofcontents
\end{frame}

\section{Introduction}
\subsection{Magic State Distillation Overview}
\begin{frame}{Magic State Distillation Overview}
  \begin{itemize}
    \item Placeholder overview of why magic state distillation is required for fault-tolerant quantum computing.
    \item Outline key challenges addressed by distillation in this study.
  \end{itemize}
\end{frame}

\subsection{Transversal Gatesets Background}
\begin{frame}{Transversal Gatesets Background}
  \begin{itemize}
    \item Placeholder summary of transversal gate properties and their role in error mitigation.
    \item Notes on constraints imposed by transversal operations on encoded logical gates.
  \end{itemize}
\end{frame}

\section{Background}
\subsection{Stabilizer Formalization}
\begin{frame}{Stabilizer Formalization}
  \begin{itemize}
    \item Placeholder description of stabilizer state notation and generators.
    \item Brief reminder of how stabilizers define code spaces.
  \end{itemize}
\end{frame}

\subsection{CSS Code Formalization}
\begin{frame}{CSS Code Formalization}
  \begin{itemize}
    \item Placeholder notes on constructing CSS codes from classical linear codes.
    \item Mention error detection and correction properties relevant to distillation.
  \end{itemize}
\end{frame}

\subsection{Self-Dual Codes}
\begin{frame}{Self-Dual Codes}
  \begin{itemize}
    \item Placeholder explanation of self-duality criteria in the CSS setting.
    \item Comments on why self-dual structures matter for transversal gate compatibility.
  \end{itemize}
\end{frame}

\section{Overview of Selected Code Family}
\subsection{Code Structure and Construction}
\begin{frame}{Code Structure and Construction}
  \begin{itemize}
    \item Placeholder summary of the code family described in \texttt{https://arxiv.org/pdf/2408.12752}.
    \item Highlight key construction steps and parameters to emphasize during the talk.
  \end{itemize}
\end{frame}

\section{Distillation Protocols}
\begin{frame}{Distillation Protocols}
  \begin{itemize}
    \item Placeholder outline of the protocols evaluated with the selected codes.
    \item Notes on assumptions, inputs, and targeted logical states.
  \end{itemize}
\end{frame}

\section{Analysis of Selected Protocols}
\subsection{Bravyi-Haah Protocol}
\begin{frame}{Bravyi-Haah Protocol}
  \begin{itemize}
    \item Placeholder recap of triorthogonal code requirements and procedure.
    \item Points to connect the protocol to the chosen code family.
  \end{itemize}
\end{frame}

\subsection{Protocol Modifications}
\begin{frame}{Slight Modifications}
  \begin{itemize}
    \item Placeholder list of tweaks made to the standard protocol.
    \item Rationale for modifications and anticipated impact on performance.
  \end{itemize}
\end{frame}

\section{Distillation Results and Yield}
\subsection{Definition}
\begin{frame}{Distillation Yield Definition}
  \begin{itemize}
    \item Placeholder definition of yield metrics used for comparison.
    \item Clarify normalization and resource accounting assumptions.
  \end{itemize}
\end{frame}

\subsection{Scaling With Code Length}
\begin{frame}{Scaling With Code Length}
  \begin{itemize}
    \item Placeholder observations on how yield changes with increasing code size.
    \item Notes on asymptotic behavior and finite-size considerations.
  \end{itemize}
\end{frame}

\section{Simulation Results}
\subsection{Simulation Process}
\begin{frame}{Simulation Process}
  \begin{itemize}
    \item Placeholder description of the simulation setup and noise models.
    \item Mention computational tools or libraries anticipated for the study.
  \end{itemize}
\end{frame}

\subsection{Contextualizing Results}
\begin{frame}{Contextualizing Results}
  \begin{itemize}
    \item Placeholder interpretation of simulated performance metrics.
    \item Discussion points comparing outcomes to expectations or baselines.
  \end{itemize}
\end{frame}

\section{Conclusion}
\subsection{Initial Hypothesis}
\begin{frame}{Initial Hypothesis}
  \begin{itemize}
    \item Placeholder statement of the working hypothesis before analysis.
    \item Criteria used to judge success or failure.
  \end{itemize}
\end{frame}

\subsection{Comparison With Results}
\begin{frame}{Comparison With Results}
  \begin{itemize}
    \item Placeholder notes contrasting the hypothesis with observed outcomes.
    \item Emphasize agreements, surprises, and unresolved questions.
  \end{itemize}
\end{frame}

\subsection{Outcome Analysis}
\begin{frame}{Outcome Analysis}
  \begin{itemize}
    \item Placeholder summary of key lessons learned from the study.
    \item Observations about robustness, efficiency, and practicality.
  \end{itemize}
\end{frame}

\subsection{Potential Future Work}
\begin{frame}{Potential Future Work}
  \begin{itemize}
    \item Placeholder list of follow-up experiments and protocol refinements.
    \item Suggestions for code design or distillation strategy improvements.
  \end{itemize}
\end{frame}

\end{document}
